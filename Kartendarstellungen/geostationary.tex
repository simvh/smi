\subsection{Geostationary Projection}
\label{sec:geostat}
In der geostationären Projektion wird die Erde aus der Perspektive eines geostationären Satellieten.
Vorteil:\newline \begin{itemize}
                  \item Wenn die Position des Satellieten bekannt ist, kann man dessen
                  Bilder als Hintergrund verwenden (siehe \ref{backgroundbilder})
                 \end{itemize}

Nachteil:\newline \begin{itemize}
                  \item Die andere Seite der Erde wird nicht dargestellt.\\
                  \item Entfernungen zwischen 2 Punkten werden auf Kreisbögen gemessen.
                 \end{itemize}

%\includegraphics[]{geostat.jpeg}