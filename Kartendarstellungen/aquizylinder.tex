\subsection{Äquidistante Zylinder Projektion}
\label{sec:aequizylinder} 
Die äquidistante Zylinder Projektion ist die einfachste Projektion. Sie stellt die Erde einfach in Längen- und Breitengrad dar. Dabei entsteht ein gleichmäßiges Gitterraster. Die Projektion ist weder winkel- noch flächentreu, das heißt, dass die Verzerrung mit der Entfernung vom Mittelpunkt der Karte zunimmt.\\ 
Vorteil der äquidistanten Zylinder Projektion:
\begin{itemize}
\item Die Projektion ist sehr einfach zu berechnen.
\end{itemize}
Nachteil der äquidistanten Zylinder Projektion:\\
\begin{itemize}
\item Die Verzerrungen wirken sowohl auf die Fläche als auch auf die Abstände aus.
\end{itemize}

Formel:\\
\begin{eqnarray}
\cal{X} & = & \lambda\\
\cal{Y} & = & \varphi
\end{eqnarray}