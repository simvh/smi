\subsection{Karte speichern}
\label{sec:save}
Um die Karte die man erstellt hat zu speichern bedient man sich des Moduls \textsf{pyplot}. Von diesem Modul kann man die Funktion \textsf{savefig(fname, dpi=None, facecolor='w', edgecolor='w', orientation='portrait', papertype=None, format=None, transparent=False, bbox\_inches=None, pad\_inches=0.1, frameon=None)} benutzen um die Figur zu speichern. Dazu gehören auch die Farblegende und der Titel der Figur. Es wird also nicht nur die Karte gespeichert. Mit dem Parameter \textsf{fname} wird der Dateiname angegeben in den gespeichert werden soll. Mit dem Parameter \textsf{format} kann ein unterstütztes Format angegeben werden, diese können variieren, meistens werden die Formate \textsf{png, pdf, ps, eps, svg} unterstützt. Die Parameter \textsf{papertype} und \textsf{orientation} werden eventuell nur für das \textsf{ps} Format unterstützt.
\subsection{Karte anzeigen}
\label{sec:show}
Mit der \textsf{pyplot} Funktion \textsf{show()} kann man die Karte anzeigen lassen. In der Anzeige kann man dann auch zoomen und die Karte verschieben. Was es einem ermöglicht auch Zeichnungen außerhalb der eigentlichen Karte zu sehen.